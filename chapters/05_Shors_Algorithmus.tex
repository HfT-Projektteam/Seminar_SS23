% Einleitung Primfaktorzerlegung\\
Die Faktorisierung großer Zahlen \cite[S189]{bernsteinPostquantumCryptography2017} ist ein fundamentales mathematisches Problem, bei dem eine gegebene Zahl in ihre Primfaktoren zerlegt wird. Dieses Problem ist von zentraler Bedeutung für die Kryptographie, da viele asymmetrische Verschlüsselungsverfahren, wie beispielsweise RSA, auf der Schwierigkeit der Faktorisierung großer Zahlen beruhen.

Das Problem der Primfaktorzerlegung besteht darin, eine öffentliche Zahl N in zwei oder mehr geheime Primzahlen p und q zu zerlegen: N = pq. Für kleine Zahlen kann die Faktorisierung durch Ausprobieren möglicher Primfaktoren relativ einfach sein. Allerdings wird die Faktorisierung bei großen Zahlen exponentiell schwieriger, da es keine effizienten klassischen Algorithmen gibt, welche dies in Polynomialzeit bewältigen können.

Die Sicherheit vieler asymmetrischer Verschlüsselungsverfahren, wie beispielsweise das RSA-Verfahren, beruht auf der Schwierigkeit der Faktorisierung großer Zahlen. Wenn ein Angreifer in der Lage wäre, die Primfaktoren einer Zahl zu finden, könnte er den geheimen Schlüssel einer Verschlüsselungsmethode berechnen und die Sicherheit des Systems kompromittieren.\\

% Shors Algorithmus
Shors Algorithmus \cite{shorAlgorithmsQuantumComputation1994} gilt als einer der bedeutendsten Quantenalgorithmen, der die Faktorisierung großer Zahlen in Polynomialzeit ermöglicht. Auch wenn dieser Algorithmus einen Quantencomputer mit vielen, stabilen Qubits voraussetzt, stellt diese revolutionäre Entdeckung bereits heute eine enorme Bedrohung für die moderne Kryptographie dar.
Zum einen wird effektiv an CQ geforscht, die Anzahl stabiler Qubits in QC steigen. Wenn der Quantencomputer seinen Durchbruch in unserer Gsellschaft erreicht hat, müssen wir darauf vorbereitet sein und uns bereits heute mit den Konsequenzen auseinander setzen. Zum anderen werden verschlüsselten Daten bereits heute gespeichert um sie in der Zukunft entschlüsseln zu können und im Nachinein an wichtige, vertrauliche Informationen zu gelangen. In der Kryptographie spricht man von dem Prinzip: "Safe now, decrypt later".\\

Shors Algorithmus bietet eine effiziente Lösung für die Faktorisierung großer Zahlen auf einem Quantencomputer. Dieser Quantenalgorithmus...

\subsection{Faktorisierung großer Zahlen}

%
Die teuerste Operation \cite*[S.]{gidneyHowFactor20482021}