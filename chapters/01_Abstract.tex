\begin{abstract}
    Quantenkryptographie wird als ernsthafte Gefahr für die aktuell gängigen Kryptographie-Verfahren gesehen. Speziell die, für das sichere Funktionieren der Netzwerk-Kommunikation benötigten, asymmetrischen Kryptographie-Algorithmen wie beispielsweise \textit{RSA} stehen in der Gefahr mithilfe eines Quantencomputers innerhalb von Stunden gebrochen zu werden. Ein solcher Computer birgt somit eine Gefahr für die gesamte heutige Informationssicherheit.\\
    Schon 1994 bewies der amerikanische Mathematiker \textit{Peter Shor} in der Theorie, wie mithilfe eines Quantencomputers die Primfaktorzerlegung großer Zahlen in realer Zeit erfolgen kann. Mit einem klassischen, binären Computer wäre dies nicht in der Lebenszeit des Universums möglich, was zentrale Sicherheits-Prinzip asymmetrischer Kryptographie darstellt.
    Wegen dieser Bedenken hat das US National Institute of Standards and Technology (NIST) seit 2016 eine Ausschreibung zur Entwicklung eines quantensicheren Algorithmus aufgestellt. Der aktuelle Favorit, ein Verfahren auf basis mehrdimensionaler Vektorfelder, soll, soweit sich keine Schwachstellen herausstellen, schon in den nächsten Jahren in der Netzwerk Kryptographie etabliert werden und somit die Kommunikation schon vor der Existenz eines potentiellen Quantencomputers absichern. Es ist also gut möglich, dass wir die aktuellen Kryptographie Standards innerhalb der nächsten 5 Jahre austauschen werden.
\end{abstract}