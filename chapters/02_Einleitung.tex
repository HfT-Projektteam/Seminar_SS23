Das Thema der \textit{Quanten Supremacy}, der Zeitpunkt zu dem ein \ac{qc} die Fähigkeit besitzt komplexe Probleme besser zu lösen als ein klassischer Computer, ist in den letzten Jahren immer wieder in einschlägigen Medien und diverse Fachpublikationen aufgetaucht. Google zum Beispiel behauptete schon wiederholt einen solchen \ac{qc} zu besitzen, was jedoch in diversen Publikationen bezweifelt wurde \cite{cho_ordinary_2022}. Wenig Zweifel besteht jedoch, dass ein solcher Computer in absehbarer Zeit einsatzbereit sein wird.\\ 
%% mabybe number of qubits to usability? 
Im Folgenden wird beschrieben, wie ein \ac{qc} funktioniert und mit welchen Algorithmen, die in realer Zeit laufen, \ac{qc} die aktuellen, kryptographischen Verfahren gebrochen werden können. Außerdem werden wir neu entwickelte, quantensichere Algorithmen betrachten und wie sie die Gefahr durch \ac{qc} mitigieren werden.